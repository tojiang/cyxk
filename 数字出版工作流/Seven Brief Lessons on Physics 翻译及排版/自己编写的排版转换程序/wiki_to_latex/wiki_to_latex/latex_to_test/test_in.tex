\documentclass[12pt,a4paper,twoside]{ctexrep}
\begin{document}

===第一章===

===最美丽的理论===

{{注释开始}}
    阿尔伯特·爱因斯坦曾在其年轻的时候,有整整一年都在漫无目的地游手好闲。其实啊有的时候,如果人们不“浪费”些时间的话,就无法获得任何东西——不幸的是,是青年人的父母常常会遗忘这一点。爱因斯坦那时在[[帕维亚]]与其家人在一起,这时的他由于德国严苛的教学,她迫不得已中断了学业。那时正值二十世纪初,意大利的工业革命正处在初级阶段。他的父亲是一名工程师,正在潘丹(paduan)平原建立第一台电力工厂。那时的阿尔伯特正在阅读康德的著作,有时还会旁听帕维亚大学的课程:他认为这只是为了开心,因为完全不需要在那里登记或者为考试而担惊受怕。就是在这样的环境下下,一位严谨的科学家诞生了。
{{注释|注释=
*译者注1:事实上彼时爱因斯坦并没有在德国有一个良好的学习环境,他讨厌那里军国主义的氛围,同时也被老师和同学视为异类,据其自己陈述,:“德国政府过分强调军国主义精神 , 这与我是格格不入的 ,甚至当我还是个孩子的时候就是如此 。我父亲迁居到意大利后 , 在我的请求下 ,他采取措施 ,使我放弃了德国国籍 , 因为我想成为一个瑞士公民 。”《纪念爱因斯坦译文集》 , 赵中立 、许良英编译 , 上海 :上海科学技术出版社 , 1979 年第 1 版 , 第 101~ 102 页 。
}}
    在登记入学[[苏黎世联邦理工学院]]后,他沉浸于物理学的世界中。几年后,1905年,他在当时最为著名的科学杂志,[[wikipedia:Annalen der Physik|Annalen der Physik]]上发表了三篇文章。这三篇文章每一篇都是诺贝尔奖级别的文章。第一篇文章揭示了原子的存在性。第二篇文章为量子力学奠定了最初的基础,我将在下一章详细阐述。第三篇文章提出了他关于[[相对论]]最初的理论(今天称之为[[狭义相对论]]),这一理论揭示了对于不同的人来说,时间并不是以相同的速度流逝的:如果其中一人以很快的速度移动,那么两个[[孪生子]]会发现他们年龄不同。
    爱因斯坦一夜之间成为了一位著名的物理学家并得到了许多高校的聘书。但是还是有一些事情使他烦躁:尽管这一理论产生了轰动效应,但它并没有与我们所知的引力发生联系,即物理下落的原因。在他为他的理论写一篇总结时,他意识到了这一点,并开始思考由现代物理学之父牛顿所提出的引力公式是否需要修改,以使其与新的相对性观念相吻合。他埋头于解决这一艰涩的问题,十年光阴悄然而逝。这十年的时间里充满了枯燥的学习,尝试,错误,混乱以及错误的文章,新奇的想法和被误解的观点。最终,在1915年的11月,他呈递了一份彻底解决这一问题的文章:一个关于引力的新理论,他称之为[[广义相对论]]——这是他的杰作,被伟大的俄罗斯物理学家[[wikipedia:Lev Davidovich Landau|列夫·朗道]]称为“最美丽的理论”。
{{注释|注释=
*译者注2:从某种程度来讲,狭义相对论是为了解决电动力学的非自洽问题,而广义相对论是为了解决引力问题。
}}
    就像[[wikipedia:Wolfgang Amadeus Mozart|莫扎特]]的[[安魂曲]];荷马的[[奥德赛]];[[西斯廷教堂]];[[李尔王]]诸如此类的许许多多的杰作常常使我们有震撼心灵的感动。去完全领略它们的优美之处往往需要长时间的咀嚼沉淀,但是这种美的奖赏易于让人陶醉其中——不仅如此,他们还开辟了一种我们欣赏世界的全新视角。至于爱因斯坦的宝石,广义相对论,这是描述自然界秩序的典范之作。
    我还记得我刚开始理解广义相对论时的兴奋与激动。记得那是一个夏天,我正在卡拉布里亚的Condofuri的海滩上沐浴着希腊地区地中海的阳光,那时的我还有一年就要大学毕业了。假期是一个人学习的最佳时期,因为完全不用收到学业压力的干扰。我那时正捧着一本被老鼠咬破了边缘的书埋头苦读,这是因为在晚上我要用这本书去堵住一个位于Umbrian山丘上的快要荒废的破旧房子里的老鼠洞,从前我常常躲在这间小屋里来逃避一些无聊的大学课程。我总是会从书中抬起头看远方波光粼粼的大海:我好像真正看见了爱因斯坦所想象的空间和时间的弯曲。这一切好像是魔法:好像是一个朋友在低声告诉我一个非凡的秘密,突然间揭开了现实的面纱,使之显露出一个更为简单也更为深刻的秩序。自从我们发现地球是圆的并且像一个陀螺一样不断自转,我们其实就已经发现,现实并不是看上去那样的:每次我们发现现实的一个新方面的知识,我们就会感受到一次深刻的情感体验。这次,又一层面纱被我们撕掉了。
    但是历史上,在超越我们的认知的一个接着一个的发现中,爱因斯坦或许是无可匹敌的。这是为什么呢?
    首先,一旦你理解了相对论,你就会发现它那优美的简洁性。我对它做了这样的总结。
    牛顿尽其最大的努力去解释物体要掉落和行星要公转的原因。他设想了一个使得所有物体聚集在一起的力,称之为[[引力]]。这个力在距离很远但是中间没有其它物体的物体之间发生作用的途径还是未知的——而且牛顿很谨慎地给出了一个假设。他想象物体在空间中移动,空间本身像一个非常大的空容器,大到将整个宇宙都囊括进入,在这个容器中的所有物体都各行其是,直到有力使它们的轨迹发生弯曲。然而对于牛顿发明的这个,包括整个世界的容器——空间,它具体由什么组成的呢?他自己也说不清楚。但是爱因斯坦出生后几年,两位伟大的英国物理学家[[wikipedia:Michael Faraday|迈克尔·法拉第]]和[[wikipedia:James Clerk Maxwell|詹姆斯·麦克斯韦]]向牛顿的经典力学体系中添加了关键性的一部分:那就是——[[电磁场]]。这个场是一个真实的实体,充斥空间,传播[[电磁波]],可以像湖面一样[[振动]],借此传递[[电磁力]]。由于爱因斯坦年轻的时候就被他父亲所建立的发电厂中驱动转子运动的电磁场所吸引,所以他不久后就理解了引力,像电场力一样,一定是被一个场所传播的:一个类比于电磁场的[[引力场]]一定是存在的。他的目标是理解引力场的工作规律,以及用数学方程式描述它。
    在那时,他想到了一个非凡的完全是天才才有的想法:引力场并不是充斥于空间的,引力场就是空间本身。这就是广义相对论的雏形。牛顿的空间,就是那个物体移动所经过的空间,以及引力场是完全相同的一个事物。
    这是一个富有启迪的时刻。一个对于世界极为重要的简化:空间不再是和物质无关的事物,它也是一种组成这个世界的物质。它是一个可以弯曲的事物。我们并不是被装在一个僵硬的看不见的结构中:我们处于一个非常大的弹性蜗牛壳中。太阳使得它周围的空间弯曲,而地球也不是因为一种神秘的力才绕着太阳公转,它是因为它向太阳笔直前进,像一块大理石在漏斗表面滚动一样。在漏斗中央并不存在一个维持这一切的神秘的作用力;是被弯曲的空间导致了大理石的滚动。行星的绕日公转,以及物体的下落,都是因为空间被弯曲了。
{{注释|注释=
*译者注3:作者在这里描述得十分简单,但简单的原理背后蕴含着复杂的数学推演与逻辑判断,这需要高深的数学和物理学知识,如果读者希望能深入了解这一部分内容,建议优先学习黎曼几何等数学知识,打好基础后再对广义相对论作进一步研究,以避免直接接触高深的物理学知识而无所适从。
}}
    我们怎样才能描述空间的曲率呢?十九世纪最为杰出的数学家,有“数学王子”之称的[[wikipedia:Johann Carl Friedrich Gauss|卡尔·弗里德里希·高斯]],已经提出了描述二维曲面,例如山丘表面,的数学方程式。之后,他让一个聪明的学生去将这一理论推广至三维乃至更多维空间中。回答这一问题的学生,[[wikipedia:Georg Friedrich Bernhard Riemann|伯纳德·黎曼]],提出了一种对这一问题影响深远的基本原则,尽管看上去它完全是无用的。黎曼论述的结论是弯曲空间是被特定数学实体承载,这一数学实体我们今天称之为[[黎曼曲面]],通常用R来标记。爱因斯坦提出了一个方程式,在这一方程式中R等价于物体的能量。这就是说空间在有物质存在的地方会发生弯曲。结论就是这样。(The equation fits into half a line, and there is nothing more)如果说这一等式解决了一半的问题,那么这里就没有其它的问题了。一个空间弯曲的观测结果变成了一个等式。

{{注释|注释=
*译者注4:事实上高斯已经提出了计算空间曲率的一般方法,只是没有完全搭建起来后面我们所称的黎曼几何的数学体系。
}}

    但是,这一个方程式描述了整个宇宙。这一理论的丰富性在这里打开一连串幻觉似的的预言,这些预言像极了疯子的的胡言乱语,但是这些在之后都被证明是真的。
    首先,这一方程式描述了星体周围的空间是怎样弯曲的。因为曲率的存在,不仅仅行星绕着恒星公转,而且光线也不再沿着直线传播而是发生一定的偏转。在1919年,这一偏转被测量了出来,结果和理论符合的很好。但不仅仅空间会弯曲,时间也会弯曲。爱因斯坦预言在地球表面,相较于低处而言,在高处时间会流逝地更快。这也被实验所证实了。如果一个住在海平面高度的人和他住在山上的孪生兄弟相遇,他会发现他兄弟会比他稍稍老一点。而这才是刚刚开始。
    当一个大的星体烧尽了它所有的可燃烧的物质(氢)时,它就会坍塌。当剩余的部分不再能承载燃烧的热时,它就会由于自身的重量坍塌,直到它使得空间坍塌到产生一个真实的洞。这就是著名的[[黑洞]]。当我在大学学习这些时,它们仅仅被认为是这一复杂理论的可能预言。今天,他们被数以百计地观测出来,并被天文学家们仔细地研究。
    但这依旧不是全部内容。这个空间都可以膨胀或者收缩。而且,爱因斯坦的方程式揭示了空间不可能是静态的,它一定是在膨胀。在1930年,宇宙的膨胀被真正地观测出来。同样的一个方程预言这一膨胀应该是被一个年轻的非常小但非常热的宇宙爆发所引起的:现在我们称之为“[[wikipedia:Big Bang|大爆炸理论]]”。又一次,最初没有人相信这一理论,但是相关的证据一直积累到[[宇宙背景辐射]]——由于最初爆炸所产生扩散的微波——被直接观测到。爱因斯坦方程式所产生的预言又一次被证明是正确的。但是进一步,这一理论断言空间像大海的表面一样运动。这些引力波的影响在宇宙中的双星系统中被观测出来,又一次与这一理论的预言相吻合,甚至吻合到令人惊讶的一千亿分之一的数量级上。其它情况也是这样。
{{注释|注释=
*译者注5:作者没有提到,这一千亿分之一事实上也是物理学关注的一个焦点:如果只有所有参数都恰到好处才会产生我们这样的宇宙,那是否说明我们这样的宇宙有些过于特殊了呢?
}}
    简而言之,这一理论描述了一个多姿多彩并且令人惊奇的世界。在这里,宇宙爆炸,空间坍塌成为一个无底洞,时间在行星附近变得缓慢,无边无际的星际空间像大海表面一样波动……这所有的一切,都渐渐地在我那本被老鼠咬过的书中出现,但它们并不是一个精神错乱的白痴所讲的故事,也不是一个被卡拉比亚地区地中海的烈日和令人眩晕的大海所引发的幻觉。这一切都是现实。
    或者更好地说,现实的一小部分,只是现实所蒙上诸多面纱中的一小部分。这个现实似乎是由和构成我们梦境相同的事物所构成的,但是比我们朦胧的梦境更为真实。
    所有的这一切都是一个基本的直觉的结果:空间和引力场是同样的一个东西。尽管你们几乎一定不能去理解这一简单的方程式,但我还是要在这里写下它。或许任何读到这的人都能去欣赏它那完美的简洁性:
    
                            [[File:爱因斯坦场方程.jpg]]
    就是这样。
    你或许,当然,为了掌握阅读和使用这一方程式需要去学习和理解黎曼的数学工具。这会耗费一些时间和精力。但是相对于欣赏贝多芬后期弦乐曲秘密的美丽,这都是次要的。在这两种情形下,奖赏都是纯粹的美丽,以及看待世界的一种崭新视角。
{{注释结束}}


===索引===
*[[Seven Brief Lessons on Physics|主页]]
*[[《Seven brief lessons on physics》--序言|序言]]
*[[《Seven brief lessons on physics》--最美丽的理论|第一章:最美丽的理论]]
*[[《Seven brief lessons on physics》--量子|第二章:量子]]
*[[《Seven brief lessons on physics》--宇宙的结构|第三章:宇宙的结构]]
*[[《Seven brief lessons on physics》--颗粒|第四章:颗粒]]
*[[《Seven brief lessons on physics》--粒空间|第五章:粒空间]]
*[[《Seven brief lessons on physics》--概率、时间与黑洞的热|第六章:概率、时间与黑洞的热]]
*[[《Seven brief lessons on physics》--我们自己|第七章:我们自己]]
\end{document}