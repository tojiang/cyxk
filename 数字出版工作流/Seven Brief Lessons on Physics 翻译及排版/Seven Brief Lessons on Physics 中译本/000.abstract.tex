	\chapter*{译者序}
	\indent

	

    《七堂极简物理课》是一本由意大利物理学家Carlo Rovelli执笔的广受欢迎的科普读物。
	
	Seven Brief Lessons on Physics is a popular science book written by Carlo Rovelli, an Italian Physicist.

	这本书也被清华大学“超越学科的认知基础”课程所选为编辑wiki的练习对象。其练习目的除了将英文翻译成中文,更重要的是基于译者对文章内容的理解以及他们自己的认知观进行必要的注释。
	
	This book is also used as an exercise for wiki editing for the Cognitive Foundations course at Tsinghua. The purpose is not just to create an English to Chinese translation, but also to annotate the essence of the book based on translators' interpretation.

	



   《七堂极简物理课》是一本很有趣的书

    作者通过对现代物理学许多基本概念的阐述,为读者们构建现代物理学的框架。在保持语言严谨的前提下,又生发以简洁的描述带领读者朋友们走进物理的殿堂。从科学普及的角度来讲,这本书达到了普及物理学的基本要求;从传授知识的角度来讲,虽然许多概念作者并没有给它们以严谨而准确的定义,但是这本书可以成为那些对这些感兴趣的人入门的导引。就像在最美丽的理论那一章所描述的,如果读者对这个问题感兴趣的话,可以自己学习黎曼几何的有关知识,进而对广义相对论的数学要求有一个基本的把握,这之后就可以进行广义相对论的学习。从这个角度来讲本书也达到了传授知识的目的。

    同时,考虑到我们现在这门课的背景,这本书更显得难能可贵——在升华概念的过程中,我们所要学习的不仅仅是概念本身,更多的是人类认识这些概念的过程与方法,这才是认知的核心。而且物理学作为十分艰深的一门学科,有着非常好的代表性与导引性。从这个角度而言,这本书对于我们把我认知的过程,了解认知的细节,都有着十分重要的作用。

                                                                                               ——第一组翻译小组


    《七堂极简物理课》是一本非常值得阅读的书。

    虽然书本身只是由曾刊登在报刊上的七篇小短文组成,但是具有着不一般的价值。首先,这种价值体现在对物理知识的传播上。本书以通俗易懂的语言,向读者介绍了当代最前沿的物理知识,因其使用了大量的比喻等手法,语言生动、讲解深入浅出,作为科普读物广受欢迎。

    其次,这种价值体现在对跨学科认知的启示上。许多跨学科认知的概念都来自物理学,本书内容本身就为跨学科认知的发展提供了良好的语言和知识素材。再者,本书广泛使用本体、空间和系统的隐喻来达到帮助读者跨学科认知的手法十分值得借鉴和学习。可以说,本书一定程度上是一个跨学科认知的成功范例。

    同时,作为“超越学科的认知基础”这门课的同学们的第一个数字流出版物,每名同学都有着很多的付出。从阅读到翻译,再到整理出版,每组成员都有着一定的贡献。如今,这本书的成功出版,令所有人都感到喜悦,同时也坚定了最终制作完全属于自己的数字流出版物的信心。

    不可否认的是,恰恰因为本书语言的简约性,对各个物理概念的说明不够全面、深入。而作为大学生的译者们经验甚浅,难免会有文不达意、粗陋错误之处,还望大家能积极批评指正!

                                                                                               ——第二组翻译小组

    《七堂极简物理课》是一本深入浅出、风趣幽默的科普书。

    本书讲述了近现代的物理学理论的发展,从相对论到量子力学再到整合相对论与量子力学的圈量子引力论等等。提出一个又一个引人深思的问题,又逐一以现有理论进行解答。留给读者思考空间又以现代理论加以阐释,使人很快了解现代物理学的发展方向以及发展近况。

    书中语言风趣幽默,大量使用[[隐喻]]的手法,用人们生活中的常见现象来比喻复杂难懂的物理学概念,易于人们理解接受。同时其又从人物,科技,结构三方面,清晰的给出了物理学的架构。译者认为这不光是一本良好的科普读物,也是应用认知学写作的书籍的良好示范。

    我们的工作在于将本书翻译成了中文,并按照我们的理解添加了很多注释。在翻译文章的同时,我们也在不停的改变着我们对于这世界的认知。本书翻译过程中,融入了多位译者学习到的关于认知学的思想和知识。对于隐喻的理解,以及对于范畴论思想的融汇,使得翻译变得更有认知学思想。这部书不单单是一本物理学的科普书籍。我们更希望向读者呈现出其独特的认知观。

                                                                                               ——第三组翻译小组




	\noindent