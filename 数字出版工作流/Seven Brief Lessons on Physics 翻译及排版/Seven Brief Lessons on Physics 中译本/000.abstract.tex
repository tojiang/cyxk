	\chapter*{译者序}
	\indent

    《七堂极简物理课》是一本非常值得阅读的书。

    虽然书本身只是由曾刊登在报刊上的七篇小短文组成,但是具有着不一般的价值。首先,这种价值体现在对物理知识的传播上。本书以通俗易懂的语言,向读者介绍了当代最前沿的物理知识,因其使用了大量的比喻等手法,语言生动、讲解深入浅出,作为科普读物广受欢迎。

    其次,这种价值体现在对跨学科认知的启示上。许多跨学科认知的概念都来自物理学,本书内容本身就为跨学科认知的发展提供了良好的语言和知识素材。再者,本书广泛使用本体、空间和系统的隐喻来达到帮助读者跨学科认知的手法十分值得借鉴和学习。可以说,本书一定程度上是一个跨学科认知的成功范例。

    同时,作为“超越学科的认知基础”这门课的同学们的第一个数字流出版物,每名同学都有着很多的付出。从阅读到翻译,再到整理出版,每组成员都有着一定的贡献。如今,这本书的成功出版,令所有人都感到喜悦,同时也坚定了最终制作完全属于自己的数字流出版物的信心。

    不可否认的是,恰恰因为本书语言的简约性,对各个物理概念的说明不够全面、深入。而作为大学生的译者们经验甚浅,难免会有文不达意、粗陋错误之处,还望大家能积极批评指正!

   《七堂极简物理课》是一本很有趣的书。

    作者通过对现代物理学许多基本概念的阐述,为读者们构建现代物理学的框架。在保持语言严谨的前提下,又生发以简洁的描述带领读者朋友们走进物理的殿堂。从科学普及的角度来讲,这本书达到了普及物理学的基本要求;从传授知识的角度来讲,虽然许多概念作者并没有给它们以严谨而准确的定义,但是这本书可以成为那些对这些感兴趣的人入门的导引。就像在最美丽的理论那一章所描述的,如果读者对这个问题感兴趣的话,可以自己学习黎曼几何的有关知识,进而对广义相对论的数学要求有一个基本的把握,这之后就可以进行广义相对论的学习。从这个角度来讲本书也达到了传授知识的目的。

    同时,考虑到我们现在这门课的背景,这本书更显得难能可贵——在升华概念的过程中,我们所要学习的不仅仅是概念本身,更多的是人类认识这些概念的过程与方法,这才是认知的核心。而且物理学作为十分艰深的一门学科,有着非常好的代表性与导引性。从这个角度而言,这本书对于我们把我认知的过程,了解认知的细节,都有着十分重要的作用。

	\noindent