	\chapter{概率、时间与\\黑洞的热}
\indent

    除了我已经讨论过的描述世界基本组成的主要理论以外,物理学仍然有一大堡垒与其他理论有所不同。一个简单的问题意想不到地产生了它:“什么是热?”

    截止到19世纪中叶时,物理学家们尝试着把热理解为一种叫做“
\href{http://toyhouse.cc/wiki/index.php/热质}{热质}
”的流体;或者是两种流体:一种热的,一种冷的。这个想法后来被证明是错误的。
\href{http://toyhouse.cc/wiki/index.php/詹姆斯·克拉克·麦克斯韦}{詹姆斯·麦克斯韦}
与奥地利物理学家
\href{http://toyhouse.cc/wiki/index.php/路德维希·玻尔兹曼}{路德维希·玻尔兹曼}
最终理解了热的本质。他们的解释十分奇特出众而影响深远——并且带领我们进入了很大程度上尚未被探索的领域。

    他们的研究表明一个有热量的物体并不包含任何热质。它不过是由运动更剧烈的
\href{http://toyhouse.cc/wiki/index.php/原子}{原子}
们所组成。原子与
\href{http://toyhouse.cc/wiki/index.php/分子}{分子}
以及束缚在一起的小型原子团不停地运动着。它们飞驰着、振动着、碰撞着……冷空气是由运动更缓慢的原子——或者说,分子——所构成的。而热空气是由运动更激烈的分子构成的。这个理论简洁而美丽,但却不止于此。

    正像我们所熟知的那样,热量总是从热的物体传向冷的物体。比如一个放在一杯热茶中的冷茶匙会变热,而我们不在气温低时多穿一点就会很快丧失我们身体的热量并感到寒冷。那么为什么热量会从热的物体传向冷的物体而不是相反呢?
\footnote[1]
{
作者开篇引出了热是什么这一问题?显然热对于我们来说是再平庸不过的日常触手可及的东西。可具体描述其又是相当具有难度的事情。事实上本章就要告诉大家热的这些特性与时间这一物理量有着密不可分的关系。
}

    这是一个关键的问题,因为这关系到时间的特性。在热量交换不发生或不明显的每个案例中,我们都能观察到系统的将来与过去十分相似。例如,热与太阳系中行星的运动几乎完全不相干,而事实上这样的运动可以反向的方式同等发生而不违反任何物理法则。然而,只要有热的存在,未来就不会与过去相同。比如说在没有
\href{http://toyhouse.cc/wiki/index.php/摩擦}{摩擦}
的情况下,摆锤会不停地摆下去。如果我们将其拍摄下来并将录像倒放,我们将会发现这运动是完全可能的。但是如果存在摩擦,那么摆锤将会极轻微地传热给它的支撑物并失去能量、缓慢减速:摩擦生热。而且我们能立刻分辨出未来(沿摆锤减速的时间方向)与过去的不同。我们从未见过一个从静止状态开始通过从支撑物吸收热进而摆动的摆锤。未来与过去的差别只有存在热时才会出现。这种基本物理现象借助热量由热的物体导向冷的物体实现了未来与过去的分离。

    那么,再一次的,为什么随着时间流逝热从热的物体传向冷的物体而不是相反呢?


    玻尔兹曼发现了这个简单得令人惊奇的原因:纯粹的
\href{http://toyhouse.cc/wiki/index.php/概率}{概率}
。

    玻尔兹曼的想法十分微妙,他引入了概率的理论。热并不是由于一条绝对的定律决定了它是由热的物体传向冷的物体的:热有如此的表现只是因为更大的可能性。这背后的原因是当一个处在较热物体中的快速移动的原子撞到一个较冷的原子时会更有可能传递给后者些许其自身的能量而不是相反。碰撞传递了能量但每次碰撞却并不同等地分配着能量。因此相互接触的物体的温度才会趋于一致。一个较冷物体接触较热物体使较热的物体升温是绝对不可能的。
    

\href{http://toyhouse.cc/wiki/index.php/概率论}{概率论}
向物理学核心的引入与在解释热动力学的基础方面,在其初期被认为是十分荒唐的。当时经常会看到没有人认真对待
\href{http://toyhouse.cc/wiki/index.php/玻尔兹曼}{玻尔兹曼}
的理论。1906.9.5,他在Trieste附近的Duino上吊自杀,永远也没有机会目睹他的理论收到普遍的承认。
\footnote[1]
{
玻尔兹曼的原子论曾被费曼如此评价:“假如在一次浩劫中所有的科学知识都被摧毁,只剩下一句话留给后代,什么样的语句可用最少的词汇包含最多的信息呢?我相信,这就是原子假说。”由此可见原子论对人类的重要性。然而当时的人们并没有意识到这一点。这导致了抑郁的玻尔兹曼的三次自杀,前两次自杀成功被人救下,最后一次却未能有人及时相救。遗憾的是,仅仅在其自杀两年之后,1908年,法国物理学家佩兰的实验最终判定了奥斯特瓦尔德“唯能论”的失败,奥斯特瓦尔德最后公开接受原子论。然而,玻尔兹曼已经没有办法亲眼见证自己理论的胜利了。
}

    在第二课中我将
\href{http://toyhouse.cc/wiki/index.php/量子力学}{量子力学}
如何预测每一小块物体的运动通过概率联系起来。这也将概率论放进了理论中央。但是玻尔兹曼考虑的关于热的本质的概率论与之有所不同。热力学中的概率在某种意义上与我们的无知紧密相连。

    也许我并不知晓有关
\href{http://toyhouse.cc/wiki/index.php/概率论}{概率论}
的相关内容,但我仍然可以确定某事发生的较小或较大的概率。例如,我不知道明天马赛港会下雨,也不会知道是晴天还是下雪,但是马赛港8月份下雪的概率很低。同样,对于大多数物体,我们有些许了解,但不完全知晓关于它们的状态的一切,我们只能根据概率进行预测。想象一个充满空气的气球。我可以测量它:测量它的形状、体积、压强、温度……但是气球内部的空气分子在其中快速移动,我不知道它们的确切位置。这阻止了我精确地预测气球的行为。例如,如果我解开这个结的封条,让空气流出,那他就会疯狂放气,左冲右突地让我无法预测他的位置。因为我只知道它的形状、体积、压力和温度,所以永远无法预测他放气状态下的位置。气球的运动取决于我无法了解的内部分子的碰撞。然而,即使我不能准确地预测每一件事,我也能预测一件事或另一件事发生的可能性。例如,气球几乎不可能会飞出窗外绕着远处的灯塔旋转,然后降落到我的手上,回到它被释放的那一点。有些行为发生的概率更大,而有些行为几乎不可能发生。

    同样,分子碰撞时热从较热的物体传递到较冷的物体的概率可以计算出比热向热物体传递的概率大得多。

    解释这些事实的科学分支被称为
\href{http://toyhouse.cc/wiki/index.php/统计物理学}{统计物理学}
。它的一个起自
\href{http://toyhouse.cc/wiki/index.php/玻尔兹曼}{玻尔兹曼}
的成果解释了热和温度的概率性——那就是
\href{http://toyhouse.cc/wiki/index.php/热力学}{热力学}
。

\footnote[3]
{玻尔兹曼提出了一个热力学基本常量,即玻尔兹曼常数。这里所说的成果应该是玻尔兹曼提出的关于单个气体分子的平均动能随热力学温度T变化的系数。其公式可以写为Ek=(3/2)kT。<br/>式中Ek为单个分子的平均动能,T为热力学温度。
}
 
   乍看之下,我们的无知昭示了物质的属性的想法似乎并不合理:热茶中的冷茶匙变热和气球被释放乱飞时。我们知道或不知道的事实与物理定律有什么关系?这个问题是合理的;而它的答案十分微妙。

    茶匙和气球的行为必须遵循完全独立于我们认知的物理定律。它们行为的可预测性或不可预知性与它们的精确状态无关;而与其相互作用的属性的概率有关。这组属性取决于我们与茶匙或气球相互作用的具体方式。概率并不是指物质本身的进化。它涉及到与我们相互作用的那些具体数量的演变。再次地,我们用来组织世界的概念的深刻的本质关系从中显现出来了。

    冷茶匙在热茶中加热是因为茶和勺子相互作用是通过描述它们微观状态的无数的变量中的有限变量下完成的。这些变量的值是不足以准确预测未来的行为(比如气球的飞行),但足以最佳可能的预测到勺子会被加热。

    我希望读者没有丧失对这些精妙的辨别的注重……

    现在,在二十世纪
\href{http://toyhouse.cc/wiki/index.php/热力学}{热力学}
(即关于热的科学)和
\href{http://toyhouse.cc/wiki/index.php/统计力学}{统计力学}
(即不同运动的概率的科学)的课程已经扩展到了电磁和量子现象。然而,包含引力场的扩展被证明是有问题的。重力场在加热时的行为如何,仍是一个尚未解决的问题。

    我们知道在加热的
\href{http://toyhouse.cc/wiki/index.php/电磁场}{电磁场}
中会发生什么:例如,在烤箱中有一种热电磁辐射可以烹饪馅饼,我们知道如何描述它。电磁波振动,随机的辐射能量,我们将所有的这团
\href{http://toyhouse.cc/wiki/index.php/光子}{光子}
想象为在一个热气球中运动的
\href{http://toyhouse.cc/wiki/index.php/分子}{分子}
。但是什么是热引力场呢?

\footnote[4]
{
我们再清晰的解释一下引力场到底是什么。相信大家对于牛顿被苹果砸的故事早已耳熟能详对于他的万有引力理论想必也早有耳闻了。引力场,是一种客观存在于空间的矢量场。对于标量的引力势求梯度可以得到引力强度。现在空间中测量得到的引力场,与利用广义相对论算得的该有的引力场值并不相同,人们认为这是因为空间中普遍存在一种,看不见摸不到但是却有质量的暗物质,这种物质提供了额外的引力,使得我们的世界成为了现有的样子。
}

    正如我们在第一节课中看到的,引力场即是空间本身。它影响时空。因此,当热量被扩散到重力场时,时间和空间本身必须振动……但我们仍然不知道如何很好的描述这个现象。我们没有方程来描述热时空的振动。以及什么是振动的时间?

    这些问题把我们带到时间问题的核心:时间的确切流向是什么?

    这个问题已经存在于古典物理学中,哲学家们在第十九、第二十世纪强调了这一点,但它在现代物理学中变得更为尖锐。物理学用公式来描述世界,它告诉我们事物如何随时间的变化而变化。但我们可以写下告诉我们事物如何关于位置改变的函数,亦或是食物的味道关于黄油数量改变的函数。但时间似乎是流动的,而黄油或空间的位置并不流动。它们之间的差别来自哪里呢?

    提出问题的另一种方法是问自己:什么是“现在”?我们说,只有现在的事物存在:过去不再存在,未来也不存在。但在物理学中,没有东西与“现在”这个概念相对应。比较“现在”和“这里”。这里指定说话者的位置:两个不同的人在这里指向两个不同的地方。因此“这里”是一个词,它的意思是什么是取决于说出他的人在哪里。这种话语的术语是“索引”。“现在”也指说出这个词的瞬间,也归类为“索引”。但是没有人会梦想说“存在于这里”,而不存在于此的事物并不存在。那么为什么我们说“现在”存在,而其他事物不存在?“现在”是客观存在的东西吗?“流动”,使事物“一个接一个地存在”,还是只是客观的,像“这里”?

    这似乎是一个深奥的心理问题。但在现代物理学中已成为一个热点问题,因为狭义相对论表明,“现在”这个概念也是主观的。物理学家和哲学家得出的结论是,时间的概念对于整个宇宙来说就是一个幻象,宇宙的“时间流”是一个不起作用的概括。当他的意大利朋友贝斯去世时,爱因斯坦写了一封感人的信给米歇尔的姐姐:“米歇尔已经一点点在我面前离开了这个陌生的世界。这不意味什么。像我们这样相信物理的人,知道过去、现在和未来之间的区别只不过是一种固执的甚至顽固的幻觉。

    幻想与否、是什么向我们解释了时间的流动与消逝呢?时间的流逝对我们所有人来说都是显而易见的:我们的思想和我们的语言都存在于时间之中;我们语言的结构需要时间——一件事物“是”或“曾是”或“将来是”。人们可以想象一个没有色彩的,没有物质的,甚至没有空间的世界,但很难想象一个世界没有时间。德国哲学家
\href{http://toyhouse.cc/wiki/index.php/马丁·海德格尔}{马丁·海德格尔}
强调了我们是“时间的栖居”。海德格尔对原始世界的描述是否有可能在世界的描述中消失?

    海德格尔的忠实哲学追随者们得出结论:物理学不能描述现实的最基本的方面,并将其视为一种误导性的知识形式。但很多时候,在过去我们已经意识到,我们眼前的直觉并不是准确的:如果我们依然笃信于此,我们将仍相信地球是平的,且太阳是围绕地球转的。我们的直觉是在我们有限的经验基础上发展而来的。当我们向前看得更远一点时,我们发现世界并不是像我们眼中看到的一样:地球是圆的,站在地球另一边的人,头是在下面的,脚是在上面的。仅仅相信眼前的直觉,而不是理性、谨慎和的整合性的审视,那是不明智的:在某些教条的老人的观念中,他是拒绝相信他村外的美好世界与他所知道的世界是有什么不同的。
\footnote[5]
{
这确实是一个复杂而又难以理解的问题。在现代的量子力学建立的过程中,“意识”这种神奇的东西一度成为人们争论的焦点。相信最著名的理论大概是“薛定谔的猫”了。实验是这样的:一只猫被关在一个密闭无窗的盒子里,盒子里有一些放射性物质。一旦放射性物质衰变,有一个装置就会使锤子砸碎毒药瓶,将猫毒死。反之,衰变未发生,猫便能活下来。按照量子力学的理论薛定谔的猫如果不观测时,竟然是既活又死的。也就是说,通过人们的意识作用可以决定猫的状态。这看起开是绝对不可能的事情。但是不妨如此理解,我们看到的世界其实不是我们看到的世界,是我们的意识告诉我们我们看到的世界是这样,所以我们才认为我们看到的世界是这样。可见意识可能并不是想我们想象中的是个“玄学”。
}

 
   正如我们所看到的那样生动,我们对时间流逝的体验不需要反映现实的一个基本方面。但如果它不是根本性的,我们对时间流逝的生动体验,它来自哪里呢?
 
   我认为答案在于时间和热量之间的紧密联系。只有在有热量流动的时候,过去和未来才会有可察觉的区别。热与概率有关,而概率又与我们与世界其他地方的相互作用不符合现实的细节有关。时间的流动是从物理学中产生的,而不是在对事物的精确描述的背景下。它出现在统计和热力学的背景下。这可能是时间之谜的钥匙。现在这个概念不存在比这里这个概念客观存在是更加客观的,但是,世间的微观作用导致出现了在只通过传递无数变量相互作用的系统(例如,我们自己)中的可见的现象。

    我们的记忆和意识是建立在这些统计上的现象。一个假设的超感官是不会有“流动”的时间:
\href{http://toyhouse.cc/wiki/index.php/宇宙}{宇宙}
是一个单块的过去,现在和未来。但由于我们意识的局限性,我们只能觉察到世界的模糊景象,并活在时间之中。借用我的意大利编辑的话:“不明显比明显的要多得多。”从这有限的、模糊的焦点中,我们可以看到时间的流逝。这样就清晰了吗?不,当然不。这里依旧还有很多东西需要去被理解。

    时间处于
\href{http://toyhouse.cc/wiki/index.php/引力}{引力}
、
\href{http://toyhouse.cc/wiki/index.php/量子力学}{量子力学}
和
\href{http://toyhouse.cc/wiki/index.php/热力学}{热力学}
交叉点引起的混乱的问题的中心。我们仍然有一系列问题依旧不明了。如果我们可能已经开始了解量子引力论,就会发现它结合了这三个谜题中的两个,但我们还没有一个理论能够把我们世界的三块基本知识结合起来。

    解决这个问题的一个小小的线索来自于物理学家
\href{http://toyhouse.cc/wiki/index.php/史蒂芬霍金|史蒂芬·霍金}{史蒂芬霍金|史蒂芬·霍金}
所完成的计算。尽管这位物理学家以被限制在轮椅上且不借助辅助器械无法发声的身体条件而出名,他依旧持续不断地钻研着物理。
\footnote[6]
{
这里指的1974年,霍金利用量子力学认真的研究了黑洞邻近的粒子行为后宣布黑洞具有温度,就像所有具有温度的物体一样,黑洞也能产生辐射!这种现象被称为霍金辐射。这里有段小插曲,上个世纪70年代,黑洞无毛定律被接受,意味着黑洞仅有质量、电荷和角动量三个参数,继而在1973年,霍金和另外两位物理学家合作写了一篇题为《黑洞的热力学定律》的论文,总结了与热力学定律相似的一系列关于黑洞的定律。该论文中着重强调了黑洞的温度为零(由于没人任何东西可以逃脱黑洞,因此它们不会辐射),并且不具有物理熵。但是,一位年轻的研究生雅各布·贝肯斯坦并不同意这个观点。他意识到如果黑洞不具备熵,热力学第二定律就会被违反。因为那样的话,我们就可以将任意具有熵的物体扔进黑洞,因此降低了外部宇宙的总熵。因此他认为黑洞的熵必须正比于表面积,才能挽救热力学第二定理。
}
 
   霍金利用
\href{http://toyhouse.cc/wiki/index.php/量子力学}{量子力学}
成功地证明
\href{http://toyhouse.cc/wiki/index.php/黑洞}{黑洞}
总是“热的”。它们像火炉一样散发热量。这是对自然中“热空间”的第一个明确提出。从来没有人观察到这种热,因为它在目前观测到的黑洞中是微弱的。但霍金的计算是令人信服的,它已被不同方式的验算所确证,黑洞的热的现在已被普遍接受。

    黑洞的热是物体上的量子效应,黑洞是自然界的引力。是单独的空间量子,空间的基本粒子,振动分子,加热了黑洞表面并产生黑洞热。这一现象涉及到量子力学、
\href{http://toyhouse.cc/wiki/index.php/广义相对论}{广义相对论}
和
\href{http://toyhouse.cc/wiki/index.php/热学}{热学}
这三个方面。黑洞的热就像物理学的
\href{http://toyhouse.cc/wiki/index.php/罗塞塔石碑}{罗塞塔石碑}
,用三种语言书写–
\href{http://toyhouse.cc/wiki/index.php/量子}{量子}
、
\href{http://toyhouse.cc/wiki/index.php/引力}{引力}
和
\href{http://toyhouse.cc/wiki/index.php/热力学}{热力学}
。它仍在等待着解读来揭示时间的本质。

\noindent
