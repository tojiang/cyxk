	\chapter{宇宙的结构}
\indent

    在20世纪前半页,
\href{http://toyhouse.cc/wiki/index.php/爱因斯坦}{爱因斯坦}
(Einstein)描述了时间与空间的原理,与此同时,
\href{http://toyhouse.cc/wiki/index.php/尼尔斯·玻尔}{尼尔斯·玻尔}
(Niels Bohr)和他年轻的学徒醉心于研究描述物质奇异
\href{http://toyhouse.cc/wiki/index.php/量子}{量子}
现象的方程。在这一世纪的下半叶,物理学家们在这两个理论基础之上,把这两个全新的理论应用到描述自然现象的各个领域:从宇宙的宏观结构到
\href{http://toyhouse.cc/wiki/index.php/基本粒子}{基本粒子}
的微观结构。我将在这一章讲述前者,在下一章讲述后者。


    这一章的组成大部分都是简笔画。这样做的原因是,在实验,测量,数学以及严密的推理之前,科学是建立在关于视野的一切之上的。科学从视野开始。科学的思维是被从同以往不同的角度“看待”事物的能力培养出的。在不同视野间遨游,而我想在此提供这次物理学之行的简要适度的轮廓。
\footnote[1]
{
科学建立在视野之上,在这里视野指的某种可观测的对象。后面的例子以宇宙作为观测对象;而前面提到的两个科学家的视野则分别是,爱因斯坦看到的
\href{http://toyhouse.cc/wiki/index.php/电磁学}{电磁学}
中光速不变,以及玻尔看到的原子分立光谱。
}

\begin{figure}[htbp]
\begin{minipage}[t]{0.3\linewidth}
\centering
\bc
\includegraphics[width=3\textwidth]{img/31.jpg}\\[12pt]
\ec
\caption{这幅图上下的概念在原文中已经有所展示了,但是并非所有的早期宇宙观天地都是如此独立的。在东方,比如古印度的宇宙观几乎仅仅注重大地,而对于天空并未涉及过多。甚至于天空中的星体,被视为宫殿和地上的宫殿类似,只不过在比我们高的地方而已。}
\label{fig:side:a}
\end{minipage}

\end{figure}                  

    上面这张图代表了
\href{http://toyhouse.cc/wiki/index.php/宇宙}{宇宙}
在数千年之间是如何被理解的:土地在下面,天空在上面。
\href{http://toyhouse.cc/wiki/index.php/阿那克西曼德}{阿那克西曼德}
(Anaximander)在26个世纪之前完成了第一次伟大的科学革命,他把上面的
\href{http://toyhouse.cc/wiki/index.php/宇宙}{宇宙}
的图景换成了下面的样子,那时他正在试图弄清楚
\href{http://toyhouse.cc/wiki/index.php/太阳}{太阳}
,
\href{http://toyhouse.cc/wiki/index.php/月亮}{月亮}
和星星是如何绕着我们旋转的:

\begin{figure}[htbp]
\begin{minipage}[t]{0.3\linewidth}
\centering
\bc
\includegraphics[width=3\textwidth]{img/32.jpg}\\[12pt]
\ec
\caption{对于这幅图而言,最重要的是意识到大地不是平整的。在发展到类似阶段时,仍然有相当多的宇宙观认为世界仍然是被撑起来的。因为人类的足迹是非常有限的,当时人类并不知道大地有没有背面。同时大地又被证明是弯曲的,最边缘就是会无限的深渊,就像《加勒比海盗》第三部里面对世界尽头的描述类似。不过当时麦哲伦的舰队已经完成了环球航行,确证了地球是一块“浮”在空中的东西。对于我们而言,认识到大地不是无限的平面,就基本上和大地是有限的曲面画上了等号。因此这样一幅图也更方面我们理解。相反的,描述一个弯曲但是又被支撑起来的大地是很难用如此简单的图所勾勒的。}
\label{fig:side:a}
\end{minipage}

\end{figure}        

    现在天空环绕在地球的四周,而不仅仅是它的上方,
\href{http://toyhouse.cc/wiki/index.php/地球}{地球}
是悬挂在太空中没有落下的巨石。很快有人(也许是帕门尼斯(Parmenides),也许
\href{http://toyhouse.cc/wiki/index.php/毕达哥拉斯}{毕达哥拉斯}
(Pythagoras))意识到,球形是这个悬空的地球的最合理的形状,因为其所有方向都是相同的,
\href{http://toyhouse.cc/wiki/index.php/亚里士多德}{亚里士多德}
(Aristotle)构造了令人信服的科学论证,以证实地球和环绕地球的按规律运行天体的天空所共同的球形本质。这是最后得到的宇宙的图像:
\begin{figure}[htbp]
\begin{minipage}[t]{0.3\linewidth}
\centering
\bc
\includegraphics[width=3\textwidth]{img/33.jpg}\\[12pt]
\ec
\caption{这幅图仅仅表示了地球是宇宙中心的概念,而不能等同亚里士多德时代的宇宙认知。这一概念也可以总结为
\href{http://toyhouse.cc/wiki/index.php/地心说}{地心说}
。最初地心说是欧多克斯提出的。在地心说的具体内容中,
\href{http://toyhouse.cc/wiki/index.php/地球}{地球}
向外依次有
\href{http://toyhouse.cc/wiki/index.php/月球}{月球}
、
\href{http://toyhouse.cc/wiki/index.php/水星}{水星}
、
\href{http://toyhouse.cc/wiki/index.php/金星}{金星}
、
\href{http://toyhouse.cc/wiki/index.php/太阳}{太阳}
、
\href{http://toyhouse.cc/wiki/index.php/火星}{火星}
、
\href{http://toyhouse.cc/wiki/index.php/木星}{木星}
和
\href{http://toyhouse.cc/wiki/index.php/土星}{土星}
,在各自的轨道上绕地球运转,并且可以对应“七重天”的概念。图中的轨道数只有5个,以及月球和别的行星在同一条轨道上,印证着这幅图不等同于地心说,也不完全等同于亚里士多德所处时代的宇宙认知。}
\label{fig:side:a}
\end{minipage}

\end{figure}        

    如
\href{http://toyhouse.cc/wiki/index.php/亚里士多德}{亚里士多德}
(Aristotle)在他的作品《论天》中所描述的那样,这个宇宙是直到中世纪的结束保有
\href{http://toyhouse.cc/wiki/index.php/地中海文明}{地中海文明}
特征的世界图像。这是丹特(Dante)和
\href{http://toyhouse.cc/wiki/index.php/莎士比亚}{莎士比亚}
(Shakespeare)在学校学习的世界图像。
    下一次飞跃由
\href{http://toyhouse.cc/wiki/index.php/哥白尼}{哥白尼}
(Copernicus)完成。这开创了所谓的伟大的科学革命。
\href{http://toyhouse.cc/wiki/index.php/哥白尼}{哥白尼}
(Copernicus)的世界与
\href{http://toyhouse.cc/wiki/index.php/亚里士多德}{亚里士多德}
(Aristotle)的世界并没有很大的不同:
\begin{figure}[htbp]
\begin{minipage}[t]{0.3\linewidth}
\centering
\bc
\includegraphics[width=3\textwidth]{img/34.jpg}\\[12pt]
\ec
\caption{和上一条一样,这里仅仅表示概念。至于在那个时代具体认识到了那些行星,这幅图并没有严谨的考据。}
\label{fig:side:a}
\end{minipage}

\end{figure}        

    但是实际上有一个关键性的区别。吸取了在古代就已经被考虑过的想法之后,
\href{http://toyhouse.cc/wiki/index.php/哥白尼}{哥白尼}
(Copernicus)了解到并且认为我们的
\href{http://toyhouse.cc/wiki/index.php/地球}{地球}
不是
\href{http://toyhouse.cc/wiki/index.php/行星}{行星}
舞会的中心,而在那个位置的是
\href{http://toyhouse.cc/wiki/index.php/太阳}{太阳}
。我们的星球成为行星之一,高速地绕地轴和
\href{http://toyhouse.cc/wiki/index.php/太阳}{太阳}
的运转。
    我们认知的增长从未停止,随着仪器的改良我们很快认识到我们的
\href{http://toyhouse.cc/wiki/index.php/太阳系}{太阳系}
不过是其他众多太阳系的其中一个,并且我们的
\href{http://toyhouse.cc/wiki/index.php/太阳}{太阳}
不过是一颗与其他
\href{http://toyhouse.cc/wiki/index.php/恒星}{恒星}
差不多的
\href{http://toyhouse.cc/wiki/index.php/恒星}{恒星}
。一粒极微小的尘埃在一团由一千亿颗星星组成的磅礴
\href{http://toyhouse.cc/wiki/index.php/星云}{星云}
——
\href{http://toyhouse.cc/wiki/index.php/银河系}{银河系}
之中。

\begin{figure}[htbp]
\begin{minipage}[t]{0.3\linewidth}
\centering
\bc
\includegraphics[width=3\textwidth]{img/35.jpg}\\[12pt]
\ec
\caption{大概从伽利略时代起,我们就能够观察并确认我们星系只不过是银河的一小部分。}
\label{fig:side:a}
\end{minipage}

\end{figure}       

    然而,那些研究
\href{http://toyhouse.cc/wiki/index.php/星云}{星云}
——也就是群星之间发白的云状物——的天文学家十九世纪三十年代所做出的精确测量表明,我们的
\href{http://toyhouse.cc/wiki/index.php/银河系}{银河系}
也不过是,由无数
\href{http://toyhouse.cc/wiki/index.php/星系}{星系}
组成的绵延到人类最先进的望远镜所能观测到的最远处星空的巨大
\href{http://toyhouse.cc/wiki/index.php/星系团}{星系团}
中的沧海一粟。人类的
\href{http://toyhouse.cc/wiki/index.php/宇宙}{宇宙}
图景始终如一,毫不停歇地扩张着。
    以下插图并非是一副画,它是
\href{http://toyhouse.cc/wiki/index.php/哈勃太空望远镜}{哈勃太空望远镜}
拍摄的一副照片。这张照片展现了一副比以往我们用最强大的望远镜所能看到的还要深的太空图景:如果用肉眼去观测,这不过是黑色天空中极小的一块碎片。透过
\href{http://toyhouse.cc/wiki/index.php/哈勃太空望远镜}{哈勃太空望远镜}
一片极其遥远的星尘展现在眼前。图中每一个黑色的点都是一个包含了上千亿颗像我们的
\href{http://toyhouse.cc/wiki/index.php/太阳}{太阳}
这样的
\href{http://toyhouse.cc/wiki/index.php/恒星}{恒星}
的
\href{http://toyhouse.cc/wiki/index.php/星系}{星系}
。而在近几年的观测中,我们发现这些
\href{http://toyhouse.cc/wiki/index.php/恒星}{恒星}
中的大部分都有
\href{http://toyhouse.cc/wiki/index.php/行星}{行星}
绕其左右。正因如此,
\href{http://toyhouse.cc/wiki/index.php/宇宙}{宇宙}
中应当有数千万亿亿的
\href{http://toyhouse.cc/wiki/index.php/地球}{地球}
这样的
\href{http://toyhouse.cc/wiki/index.php/行星}{行星}
。而且这总发生在我们观测的任一方向上:

\begin{figure}[htbp]
\begin{minipage}[t]{0.3\linewidth}
\centering
\bc
\includegraphics[width=3\textwidth]{img/36.jpg}\\[12pt]
\ec
\caption{}
\label{fig:side:a}
\end{minipage}

\end{figure}  

    但这种无止境的一致性,并不是看上去的那样。正如我在第一节课中所解释的,空间并非平坦,而是弯曲的。我们必须想象宇宙的纹理,星系绽放发散,被像海浪般的波所推动,有时如此剧烈以至于产生了黑洞。让我们回到一幅经过绘制的图像,来展示宇宙的伟大的波皱:
\begin{figure}[htbp]
\begin{minipage}[t]{0.3\linewidth}
\centering
\bc
\includegraphics[width=3\textwidth]{img/37.jpg}\\[12pt]
\ec
\caption{这里一般生动形象的显示出来了,大概这些纵横交错的的线条就像标尺,弯曲的标尺大概就对应于弯曲的空间。}
\label{fig:side:a}
\end{minipage}

\end{figure}  

    最终,我们知道,这个布满了星系有一百五十亿年生命的巨大而有弹性的宇宙,是从一个非常热而致密的小团中诞生出来的。为表示这一景象,我们不再需要绘制宇宙,而是绘制整个历史。就像这样:
\begin{figure}[htbp]
\begin{minipage}[t]{0.3\linewidth}
\centering
\bc
\includegraphics[width=3\textwidth]{img/38.jpg}\\[12pt]
\ec
\caption{可能在很多别的科普书中,我们看到过类似“实体”的右半支图像被用于形象描述我们的宇宙膨胀状态。在这里也概莫能外,事实上我们的宇宙也确实在加速膨胀。}
\label{fig:side:a}
\end{minipage}

\end{figure}  

    宇宙开始时是一个小球,然后膨胀到现在的宇宙尺寸。这是我们目前已知的的宇宙图景,亦是我们所知最大规模的。
    还有其他的事实吗?在这以前又发生了什么?也许是的。我将在接下来的几章中讨论。那么其他类似或不同的宇宙存在吗?问题的答案,我们还不知道。




 

\noindent
