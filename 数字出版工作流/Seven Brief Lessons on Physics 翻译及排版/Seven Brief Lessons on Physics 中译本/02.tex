	\chapter{量子}
\indent

我可以很确定的告诉大家:没有人真正了解量子力学。——狄拉克

一个物理学系统——原子的任何一种结合体——什么时候才显示出“动力学的定律”(在普朗克的意义上说)或“钟表式工作的特点”呢?量子论对这个问题有一个简短的回答,就是说,

在绝对零度时。当接近零度时,分子的无序对物理学事件不再有什么影响了。这是沃尔塞.能斯特的著名的“热定理”,有时候被冠以“热力学第三定律”的美名(第一定律是能量原理,第二定律是熵原理)。——《生命是什么》


    二十世纪物理学的两大支柱——我在第一章所提到的
\href{http://toyhouse.cc/wiki/index.php/广义相对论}{广义相对论}和现在我将要阐述的[[量子力学]],不同到不能再不同了。这两个理论都告诉我们精致的自然结构事实上比看上去更为细小。但是广义相对论是一个杰出的佳作:作为由[[爱因斯坦]]独立思考到的成果,它是一个将引力场,空间和时间视为简单的连续体来处理的。而量子力学或“量子理论”,在另一方面,其得到了无比成功的实验验证并且已经产生了一些改变我们每日生活的产品(例如,我正在使用的电脑)。然而在它诞生一个多世纪后,它还是十分神秘并且难以理解。
    据说量子力学准确地诞生于1900年,之后引导了一个世纪的诸多艰深的思考。德国物理学家
\href{https://en.wikipedia.org/wiki/马克思·普朗克}{Max Planck}计算了黑体内的辐射场。为了计算,他用了一个巧妙的方法:他想象场的能量被分配在量子中,就是说能量被分为一块一块的。这一过程计算出了和实验符合的非常好的结果(因此,在一定程度上是正确的)但是却和当时所知的所有物理规律相违背。能量一直以来被认为是连续变化的,并且没有理由去把它处理成由小的结构单元组成。对普朗克本人来说,把能量处理成由有限的结构单元组成也是一种奇怪的计算方法,他自己也没有完全理解这种方法的有效性。再一次,是五年后爱因斯坦理解了能量单元是真实存在的。爱因斯坦认为光是由能量单元组成的——光单元。今天我们称它们为[[光子]]。他在他文章的介绍中写道:''我认为,如果我们假设光的能量是不连续分布在空间中,那么与[[黑体辐射]],[[荧光]],由[[紫外光]]激发的[[阴极射线]]以及其它与光产生与传播相关的物理现象都能够更好地被理解。与假设保持一致,点光源发射的光的能量在扩张的空间中是不连续分布的,而且是由有限个位于空间中某个点的能量量子组成的,他们在移动中不会被拆分,只能作为整体产生或被吸收。''
\marginpar{*译者注1:作者在这里的描述似乎不够准确,前人在计算的时候将能量设想为一块一块的,普朗克本人并没有创造这种方法,但具有革命性意义的是,前人在计算的时候认为一块一块的能量每一块最后都是趋近于0的,但普朗克认为每一块都是一个非常小的常数,并不是0。}


    这一段简洁而清晰的论断才是量子理论诞生的标志。注意到开始的“我认为”,使我想起了“我认为……”,就像达尔文介绍进化论时,或者法拉第第一次介绍关于磁场的革命性的理论时所说的。天才也是会犹豫的。
    爱因斯坦的工作最初被物理学界的同仁认为是一个杰出年轻物理学家无意义的幼稚之作。但是随后,正是因为这项工作他被授予了诺贝尔奖。如果说普朗克是量子理论之父的话,那么爱因斯坦则是养育它的人。
{{注释|注释=
*译者注2:虽然爱因斯坦获得过诺贝尔奖这件事情广为人知,但是爱因斯坦获奖原因是光电效应领域的开创性的研究,而不是像很多人所想的那样是由于狭义相对论和广义相对论的突出贡献。因为彼时相对论并没有被物理学界完全接受。
}}
    但是就像所有的后代一样,这一理论之后在爱因斯坦没有预料到的情况下自行发展。在二十世纪第二个和第三个十年,[[wikipedia:Niels Bohr|戴维·尼尔斯·玻尔]]领导了量子理论的发展。玻尔理解了原子中电子的能量只能取固定的几个值,就像光的能量一样。重要的是,电子只能在能量一定的[[原子轨道]]之间[[跃迁]],发射或吸收一个光子。这就是著名的[[量子跃迁]]。除此之外,在他位于哥本哈根的研究院内聚集了那个世纪最为杰出的青年物理学家,他们一起探索并且试图将原子世界一些令人困惑的现象澄清,最终建立一个连贯的理论。在1925年,这一理论的基本方程出现了,代替了全部的[[牛顿力学]]。
    很难想象比这更为杰出的成就了。一时间,所有事情都有意义了,你也可以计算所有的事情。例如:你是否还记得[[wikipedia:Dmitri Mendeleev|门捷列夫]]所发明的被挂在很多教室的墙上,囊括了所有宇宙中的元素,从[[氢]]到[[铀]]的元素周期表吗?为什么每个元素都被放在了特定的位置?为什么元素周期表有其特定的结构、周期?又为什么元素有这样那样的特性呢?答案就是每个元素对应着量子力学主方程式的一个特解。整个化学都由一个方程式衍生而来。
    第一个依据那些令人眼花缭乱的假设写下这个方程的人,是德国的天才青年物理学家[[wikipedia:沃纳·海森堡|沃纳·海森堡]]。
{{注释|注释=
*译者注3:海森堡率先创建了矩阵力学,随后薛定谔创建了波动力学并且证明了它们二者的等价性。我们目前所写的基本方程,大多是薛定谔方程。
}}
    海森堡认为电子并不总是存在的。它们只在有人或者仪器观察它们的时候,或者更准确地说,它们和其它事物发生相互作用的时候才存在。它们当与其它事物碰撞的时候,会在空间中以一定可计算的概率具象化。由一个轨道到另一个轨道量子跃迁是它们具象化的唯一方法:一个电子是一系列由一次相互作用到另一次相互作用的跃迁,电子并不位于任何准确的位置。它根本不在空间中。
    这就好像上帝并没有用一个加粗的线来设计现实,只是用点来描述它的轮廓。
    在量子力学中,不存在具有准确位置的物体,除了当它们飞快地和其它事物碰撞时。为了描述由一次相互作用到另一次相互作用的碰撞期间的物理过程,我们用一种不存在于真实物理空间,仅仅存在于抽象数学空间的抽象数学模型来刻画这一过程。但是更糟糕的事情发生了:这些相互作用的跃迁并不是以可预测的方式发生,而是很大程度上随机发生。我们不可能去预言电子将会在哪里重新出现,而只能去计算它会在这里或那里出现的[[概率]]。关于概率的问题深深困惑着那些认为所有事情都是由坚固、普遍并且不可动摇的物理学规律控制的物理学家们。   
    这听上去荒唐吗?这对爱因斯坦也十分荒唐。一方面,他提名海森堡获诺贝尔奖,因为意识到他理解了一些关于这世界基础的规律;另一方面,他不放过任何机会去抱怨这理论毫无意义。
    [[wikipedia:Niels Bohr Institute|哥本哈根学派]]的年轻学者们十分沮丧:爱因斯坦怎么像这样来想?他们的精神领袖,那个曾经有勇气去思考不能思考的事物的大师,现在退缩了,畏惧这他自己所引发的到未知领域的跳跃。那个认为时间并不是永恒不变的和空间是卷曲的爱因斯坦现在却认为世界并不可能这样奇异。
    玻尔耐心地把这些新理论解释给爱因斯坦。爱因斯坦否认了这一理论。他设计了思想实验去证明这一新理论是自相矛盾的:“想象一个充满光的盒子,从这里我们允许单个的光子消失一瞬间……”之后就是他最著名的例子之一,“光盒(box of light)实验。最终波尔总能找到驳斥这些否认的答案。他们之间的对话以讲座,书信以及论文等等方式持续了很多年。在交换观点的过程中,这两位伟人都需要重新思考以调整他们的光点。爱因斯坦不得不承认新理论中不存在矛盾之处。玻尔不得不承认事情并不是像他一开始想的那样简单与清晰。爱因斯坦不想在总是存在一个不依赖于相互作用而存在的客观真实这一观点上让步。同样地,玻尔也不想在建构现实的新理论的可信性上让步。最终,爱因斯坦承认量子论是一个理解世界方面巨大的进步,但仍认为事物不应该像假设一样奇怪——这理论的背后一定存在一个更为深刻和理性的解释。
    一个世纪以后,我们现在面临着相同的问题。量子力学的方程及其解被物理学家,工程学家,化学家和生命科学家广泛地应用在不同的领域。它们在所有当代技术的领域都十分有用。但它们还是十分神秘。因为它们并没有描述物理系统内究竟发生了什么,只是刻画了一个物理系统是如何影响另一个物理系统的过程。
{{注释|注释=
*译者注4:非常著名的一个思想实验是薛定谔提出的“薛定谔的猫”实验。在这一实验中,尽管我们可以描述量子的不确定态,但是与宏观物体建立联系后,宏观物体似乎也处于一种“死生叠加态”。这一问题至今没有得到合理的解释。
}}
    这意味着什么?一个系统重要的真实性是不可描述的吗?这是否意味着我们只是缺少一部分谜题?或者这是否意味着,至少对我来说,我们必须接受现实仅仅是相互作用?我们在现实领域的知识总量在不断增加,这使得我们可以做一些我们过去甚至难以想象的事情。但是这增加又提出了新的问题,新的神秘。那些在实验室使用这些理论的人继续探索不管这些问题,但是在近些年数量急剧增加的文章和会议中,物理学家和哲学家继续去研究这些问题。在它诞生一个世纪以后,什么是量子理论?是一个对于自然真实性非常深刻的尝试?还是一个巧合与事实相符的大错?或是一个不完全谜题的一部分?亦或是一个深刻的关于世界结构的但我们目前还无法理解的理论的线索?
    爱因斯坦去世后,他最伟大的对手玻尔向他致以崇敬欣赏的悼词。当几年后玻尔去世时,有人拿出了他研究过程中的一块黑板的照片。在黑板上有一幅画。画所描述的是爱因斯坦思想实验中的“充满光的箱”。到最后,还是渴望去挑战自己以获得更多的认识与理解。到最后,还是对于事物本质的怀疑。
{{注释结束}}
    二十世纪物理学的两大支柱——我在第一章所提到的广义相对论和现在我将要阐述的量子力学,不同到不能再不同了。这两个理论都告诉我们良好的自然结构比看上去更为细小。但是广义相对论是一个紧凑的佳作:作为由爱因斯坦自己思考到的成果,它是一个对于引力,空间和时间简单并且连贯的观点。而量子力学或“量子理论”,在另一方面,已经得到不相匹配的实验成功并且已经产生了改变我们每日生活的产品(例如,我正在使用的电脑)。但是在它诞生一个多世纪后,它还是十分神秘并且难以理解。

  据说量子力学是准确地在1900年诞生,几乎引导了一个世纪的很多思考。德国物理学家马克思.普朗克计算了黑体内的辐射场。为了计算,他用了一个巧妙的方法:他想象场的能量被分配在量子中,就是说能量被分为一块一块的。这一过程导致了和实验符合非常好的结果(因此,在一定程度上是正确的)但是却和当时所知的所有物理规律相违背。能量一直以来被认为是连续变化的,并且没有理由去把它处理成由小的结构单元组成。对普朗克本人来说,把能量处理成由有限的结构单元组成也是一种奇怪的计算方法,他自己也没有完全理解这种方法的有效性。再一次,是五年后爱因斯坦理解了能量单元是真实的。爱因斯坦认为光是由能量单元组成的——光单元。今天我们称它们为光子。他在他的文章的介绍中写道:我认为,如果我们假设光的能量是不连续分布在空间中,那么与黑体辐射,荧光,由紫外光激发的阴极射线以及其它与光产生与传播相关的物理现象都能够更好地被理解。与假设保持一致,点光源发射的光的能量在扩张的空间中是不连续分布的的,而且是由有限个位于空间中某个点的能量量子组成的,他们在移动中不会被拆分,只能作为整体产生或被吸收。

  这一段简洁而清晰的论断才是量子理论诞生的标志。注意到开始的“我认为”,使我想起了“我认为……”,就像达尔文介绍进化论时,或者法拉第第一次介绍关于磁场的革命性的理论时所说的。天才也是会犹豫的。

  爱因斯坦的工作最初被物理学界的同仁认为是一个杰出年轻物理学家无意义的幼稚之作。但是随后,因为同样的工作他被授予了诺贝尔奖。如果说普朗克是量子理论之父的话,那么爱因斯坦则是养育它的人。

  但是就像所有的后代一样,这一理论之后在爱因斯坦没有预料到的情况下自行发展。在二十世纪第二个和第三个十年,戴维·尼尔斯·玻尔领导了量子理论的发展。波尔理解了原子中电子的能量只能取固定的几个值,就像光的能量一样。重要的是,电子只能在能量一定的原子轨道之间跃迁,发射或吸收一个光子。这就是著名的量子跃迁。除此之外,在他位于哥本哈根的研究院内聚集了那个世纪最为杰出的青年物理学家,他们一起探索并且试图将原子世界一些令人困惑的现象澄清,最终建立一个连贯的理论。在1925年,这一理论的基本方程出现了,代替了全部的牛顿力学。

  很难想象比这更为杰出的成就了。一时间,所有事情都有意义了,你也可以计算所有的事情。例如:你是否还记得门捷列夫所发明的有这些周期和有这些特定性质的元素的结构?答案就是每个元素对应着量子力学主方程式的一个特解。化学由一个方程式产生。

  第一个依据那些令人眼花缭乱的假设写下这个方程的人,是德国的天才青年物理学家沃纳·海森堡。元素周期表,那上面列出了由氢到铀组成宇宙的所有可能的元素,因此被挂在很多教室的墙上。为什么就是这些元素被列举在这里以及为什么元素周期表

  海森堡认为电子并不总是存在的。它们只在有人或者仪器观察它们的时候,或者更准确地说,它们和其它事物发生相互作用的时候才存在。它们当与其它事物碰撞的时候,会在空间中以一定可计算的概率具象化。由一个轨道到另一个轨道量子跃迁是它们具象化的唯一方法:一个电子是一系列由一次相互作用到另一次相互作用的跃迁,电子并不位于任何准确的位置。它根本不在空间中。

  这就好像上帝并没有用一个加粗的线来设计现实,只是用点来描述它的轮廓。

  在量子力学中,不存在具有准确位置的物体,除了当它们飞快地和其它事物碰撞时。为了描述由一次相互作用到另一次相互作用的碰撞期间的物理过程,我们用一种不存在于真实物理空间,仅仅存在于抽象数学空间的抽象数学模型来刻画这一过程。但是更糟糕的事情发生了:这些相互作用的跃迁并不是以可预测的方式发生,而是很大程度上随机发生。我们不可能去预言电子将会在哪里重新出现,而只能去计算它会在这里或那里出现的概率。关于概率的问题深深困惑着那些认为所有事情都是由坚固、普遍并且不可动摇的物理学规律控制的物理学家们。   

   这听上去荒唐吗?这对爱因斯坦也十分荒唐。一方面,他提名海森堡获诺贝尔奖,因为意识到他理解了一些关于这世界基础的规律;另一方面,他不放过任何机会去抱怨这理论毫无意义。

   哥本哈根学派的年轻学者们十分沮丧:爱因斯坦怎么像这样来想?他们的精神领袖,那个曾经有勇气去思考不能思考的事物的大师,现在退缩了,畏惧这他自己所引发的到未知领域的跳跃。那个认为时间并不是永恒不变的和空间是卷曲的爱因斯坦现在却认为世界并不可能这样奇异。

  玻尔耐心地把这些新理论解释给爱因斯坦。爱因斯坦否认了这一理论。他设计了思想实验去证明这一新理论是自相矛盾的:“想象一个充满光的盒子,从这里我们允许单个的光子消失一瞬间……”之后就是他最著名的例子之一,“光盒(box of light)实验。最终波尔总能找到驳斥这些否认的答案。他们之间的对话以讲座,书信以及论文等等方式持续了很多年。在交换观点的过程中,这两位伟人都需要重新思考以调整他们的光点。爱因斯坦不得不承认新理论中不存在矛盾之处。玻尔不得不承认事情并不是像他一开始想的那样简单与清晰。爱因斯坦不想在总是存在一个不依赖于相互作用而存在的客观真实这一观点上让步。同样地,玻尔也不想在建构现实的新理论的可信性上让步。最终,爱因斯坦承认量子论是一个理解世界方面巨大的进步,但仍认为事物不应该像假设一样奇怪——这理论的背后一定存在一个更为深刻和理性的解释。

  一个世纪以后,我们现在面临着相同的问题。量子力学的方程及其解被物理学家,工程学家,化学家和生命科学家广泛地应用在不同的领域。它们在所有当代技术的领域都十分有用。但它们还是十分神秘。因为它们并没有描述物理系统内究竟发生了什么,只是刻画了一个物理系统是如何影响另一个物理系统的过程。

  这意味着什么?一个系统重要的真实性是不可描述的吗?这是否意味着我们只是缺少一部分谜题?或者这是否意味着,至少对我来说,我们必须接受现实仅仅是相互作用?我们在现实领域的知识总量在不断增加,这使得我们可以做一些我们过去甚至难以想象的事情。但是这增加又提出了新的问题,新的神秘。那些在实验室使用这些理论的人继续探索不管这些问题,但是在近些年数量急剧增加的文章和会议中,物理学家和哲学家继续去研究这些问题。在它诞生一个世纪以后,什么是量子理论?是一个对于自然真实性非常深刻的尝试?还是一个巧合与事实相符的大错?或是一个不完全谜题的一部分?亦或是一个深刻的关于世界结构的但我们目前还无法理解的理论的线索?

  爱因斯坦去世后,他最伟大的对手玻尔向他致以崇敬欣赏的悼词。当几年后玻尔去世时,有人拿出了他研究过程中的一块黑板的照片。在黑板上有一幅画。画所描述的是爱因斯坦思想实验中的“充满光的箱”。到最后,还是渴望去挑战自己以获得更多的认识与理解。到最后,还是对于事物本质的怀疑。

\noindent
